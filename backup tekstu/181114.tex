% Template authors: Bogumił Kamiński and Michał Jakubczyk

\documentclass[12pt,a4paper,twoside,openany]{book}
\usepackage[T1]{fontenc}
\usepackage[utf8]{inputenc}
\usepackage[polish]{babel}
\usepackage{graphicx}
\usepackage{times}
\usepackage{indentfirst}
\usepackage[left=3cm,right=2cm,top=2.5cm,bottom=2.5cm]{geometry}
\usepackage{natbib}
\usepackage{enumitem}
\usepackage{color}
\usepackage{soul}
\usepackage{tikz}
\usepackage{url}
\usepackage{todonotes}
\usepackage{verbatim}
\setlist{itemsep=0pt}
\setlist{nolistsep}
\frenchspacing
\linespread{1.3}
\addto\captionspolish{%
\renewcommand*\listtablename{Spis tabel}
\renewcommand*\tablename{Tabela}
}
\usepackage{titlesec}
\titlelabel{\thetitle.\quad}

\frenchspacing

\begin{document}

\begin{center}
\includegraphics[scale=0.3]{sgh_full.png}

\vspace{1cm}

% tu i dalej fbox należy usunąć i wpisać odpowiednią wartość
Studium magisterskie
\end{center}

\vspace{1cm}

\noindent Kierunek: Metody ilościowe w~ekonomii i~systemy informacyjne

\noindent Specjalność: Procedury analizy decyzji

\vspace{1cm}

{
\leftskip=10cm\noindent
Andrzej Paluszek\newline
Nr albumu: 74790

}

\vspace{2cm}

\title{Wspomaganie podejmowania decyzji o alokacji aktywów z wykorzystaniem krzywej dochodowości}
\makeatletter

\begin{center}
\LARGE\bf
\todo[inline]{\@title}
\end{center}

\vspace{2cm}

{
\leftskip=10cm\noindent
Praca magisterska\newline 
napisana w~Instytucie Ekonometrii\newline
pod kierunkiem naukowym\newline
dra Grzegorza Kolocha

}

\vfill

\begin{center}
Warszawa, \the\year
\end{center}
\thispagestyle{empty}

\clearpage
\thispagestyle{empty}
\mbox{}
% druga strona będzie pusta, ponieważ drukujemy dwustronnie
% a mbox jest po to, żeby ta strona się pokazała
% od procenta robimy komentarze
\clearpage

\tableofcontents

\clearpage

\chapter*{Uwagi techniczne, skasuj rozdział po uwzględnieniu}

Wymagane jest wgranie pracy do systemu Overleaf (\url{https://www.overleaf.com/}) i udostępnienie promotorowi z~prawem zmian.

Pytania do promotora w~tekście proszę zadawać \todo[inline]{o tak}, tj.~stosując \verb!\todo[inline]{o tak}!.

Po kropkach niekończących zdania (np.~po `np.', `tj.', `itd.', itd.) stawiamy znak tyldy, tj.~\verb!~! (żeby zmniejszyć odstęp). Wykorzystujemy także tyldę między odwołaniem i~numerem, np.~`Tabela~3' (oczywiście odwołanie robione automatycznie przez \verb!\ref{}!). I wreszcie, stosujemy tyldę po pojedynczych literach (np.~`i', `a'), bo to niełamliwa spacja i~gwarantuje brak takich liter na końcu linii.

Przykłady odwołań źrółowych:
\begin{itemize}

\item \defcitealias{marciniak2006}{Marciniaka (2006)}\citetalias{marciniak2006};

\end{itemize}

Co warto przeczytać o~\LaTeX:
\begin{enumerate}
\footnotesize % tak zmieniamy rozmiar fontu
\item \url{http://www.tex.ac.uk/ctan/info/gentle/gentle.pdf},
\item \url{ftp://sunsite.icm.edu.pl/pub/CTAN/info/lshort/english/lshort.pdf},
\item \url{http://paws.wcu.edu/tsfoguel/tikzpgfmanual.pdf},
\item \url{https://www.overleaf.com/latex/learn/free-online-introduction-to-latex-part-1}.
\end{enumerate}

\chapter{Wprowadzenie}

Celem niniejszej pracy jest sprawdzenie możliwości wykorzystania dynamicznego modelu krzywej dochodowości autorstwa Diebold i~inni (2005) dla Polski poprzez estymację parametrów tego modelu z~wykorzystaniem filtru Kalmana i~odniesienie uzyskanych wyników do potencjalnych decyzji uczestników rynku o~alokacji swoich aktywów.

\ldots

\chapter{Terminowa struktura stóp procentowych}
\label{sec:termin}

W niniejszym rozdziale przedstawione zostaną informacje dotyczące terminowej struktury stóp procentowych, jej odbiorców, zastosowanie

\section{Czym jest krzywa dochodowości}
\label{sec:krzywa}
Na wielu zarówno rozwiniętych jak i~rozwijających się rynkach w~obrocie znajduje się pewna liczba dłużnych instrumentów finansowych o~różnej stopie zwrotu i~czasie pozostałym do wykupu. W~polu zainteresowań uczestników rynku jest między innymi badanie relacji między instrumentami w~obrębie ich jednej klasy. Biorąc to pod uwagę aby dokonać analizy obligacji konieczne jest skupienie się na takim doborze rozpatrywanych papierów, aby zarówno poziom ryzyka jak i~emitent czy kraj były jednorodne. Jeżeli dane obligacje skarbowe różnią się jedynie czasem pozostałym do wykupu możliwe jest uzyskanie terminowej struktury stóp procentowych. Jej graficzną reprezentacją jest krzywa dochodowości będąca bardzo ważnym wskaźnikiem i~źródłem wiedzy na temat stanu rynku papierów dłużnych \citep{choudhry2004}.

\section{Zastosowania}
\label{sec:zast}
Krzywa dochodowości jest narzędziem, które w dosyć niezawodny sposób pokazuje w jaką stronę zmierza rynek lub co rynek przewiduje że będzie dziać się w przyszłości. Z tego powodu uczestnicy rynku papierów dłużnych są zainteresowani tym jaki jest kształt krzywej i~jaką informacje niesie on ze sobą. Najważniejsze wnioski i~zastosowania krzywej dochodowości są następujące:

\begin{itemize}
\item Ustalenie dochodów dla pozostałych instrumentów z~rynku papierów dłużnych.

Krzywa dochodowości w~głównej mierze pokazuje koszt pozyskania kapitału w~zależności od czasu do wykupu. Oprocentowanie obligacji skarbowych, począwszy od tych od najkrótszej zapadalności do tych o najdłuższej jest benchmarkiem (stopą odniesienia) dla oprocentowań wszystkich innych instrumentów dłużnych na rynku. Znaczy to tyle, że na danym rynku wszystkie obligacje innych emitentów będą wycenione wyżej dla tych samych okresów zapadalności. Ta różnica oprocentowań (spread) jest właśnie ustalana przez emitentów z~wykorzystaniem krzywej dochodowości.

\item Działanie jako wskaźnik przyszłych stóp procentowych.

Krzywa dochodowości, w~odpowiedzi na oczekiwania rynku dotyczące oprocentowań, przyjmuje zwykle pewne określone kształty. Uczestnicy rynku obligacji analizując aktualny kształt krzywej określają kierunek zmian i~wnioski z~niego płynące dotyczące rynku papierów dłużnych a~co za tym idzie stóp oprocentowań. To bardzo ważne zastosowanie krzywej dochodowości opiera się jednak w~sporej mierze na subiektywnych wnioskach uczestników rynku. Informacje zawarte w~kształcie krzywej dochodowości wykorzystywane są nie tylko w~handlu obligacjami i~zarządzaniu funduszami, ale też na przykład w~finansach przedsiębiorstwa w~procedurach oceny projektów. Oprócz tego banki centralne i~departamenty skarbu również biorą udział w~analizie krzywej dochodowości w~celu oceny przyszłych stóp oprocentowań i~spodziewanego poziomu inflacji.

\item Mierzenie i~porównywanie zwrotów z~inwestycji dla różnych czasów zapadalności.

Menadżerowie portfeli wykorzystują krzywą dochodowości w~celu oceny względnej wartości inwestycji w~zależności od czasu jej trwania. Krzywa dochodowości wskazuje jakie zwroty są możliwe dla różnych punktów zakończenia inwestycji i~dlatego jest bardzo ważna dla zarządzających portfelami o~stałej dochodowości (fixed-income), którzy mogą ją wykorzystać do określenia który punkt na krzywej oferuje najlepszą stopę zwrotu w~porównaniu do innych punktów.

\item Wskazywanie różnicy cen pomiędzy obligacjami o podobnym czasie do wykupu.

Krzywa dochodowości może być analizowana w celu wykazania które z~dostępnych na rynku papierów są wycenione powyżej lub poniżej ich teoretycznej wartości. Umieszczając obligacje w~relacji do krzywej ułatwia rozpoznanie które z~papierów powinny zostać kupione albo sprzedane czy to bezpośrednio, czy to z~wykorzystaniemm spread tradingu.

\item Wycenianie poziomu oprocentowania instrumentów pochodnych.

Wyceny instrumentów pochodnych są ściśle związane z~krzywą dochodowości. Na krótkim końcu krzywej, czyli jej początku, instrumenty takie jak FRA (forward rate agreement) wyceniane są w~oparciu o~rynkową ocenę na przykład oprocentowania trzymiesięcznych depozytów. Na długim końcu, czyli dla miejsc na krzywej gdzie czas do wykupu przekracza około rok, swap na stopę procentową (interest rate swap) jest wyceniany w stosunku do krzywej dochodowości, podobnie instrumenty hybrydowe takie jak na przykład obligacje zamienne. Wolna od ryzyka stopa procentowa będąca jednym z~parametrów wykorzystywanym przy wycenie opcji może być powiązana z~obligacjami skarbowymi czyli krzywą dochodowości \citep{choudhry2004}. 

Wykorzystanie właśnie tych własności - powiązania krzywej dochodowości dotyczącej obligacji skarbowych i~instrumentów pochodnych, a~w~szczególności swapu na stopę procentową, i~ich wycen dostępnych na rynku pozwoli w~dużej mierze przygotować część obliczeniową w~niniejszej pracy zaprezentowaną w~rozdziale~\ref{sec:wyniki}.
\end{itemize}

Podsumowując, krzywa dochodowości ma bardzo szerokie zastosowania tak na polu analiz inwestycyjnych jak i~analiz makroekonomicznych. Przedsiębiorstwa chcąc poprawnie zarządzać swoimi finansami wykorzystują ją do analizy aktualnych i~przyszłych kosztów uzyskania kapitału, natomiast skarb państwa może dzięki niej określić sposób zarządzania zadłużeniem publicznym i~finansowania potrzeb kraju. Oprócz tego, banki centralne w~swojej polityce monetarnej dokonują analizy krzywej dochodowości w~celu rozpoznania oczekiwań wysokości ustalanych stóp procentowych, a~gospodarstwa domowe dzięki niej mogą lepiej prognozować odsetki, które są zobowiązane płacić w związku z posiadanym zadłużeniem \citep{rubaszek2012}. Całość ta sprawia że krzywa dochodowości jest bardzo dobrym wskaźnikiem zbliżającej się sytuacji makroekonomicznej. 

\section{Modelowanie krzywej dochodowości}
\label{sec:modelowanie}

\section{Przegląd literatury}
\label{sec:literat}

Problem terminowej struktury stóp procentowych i~jej graficznej reprezentacji czyli krzywej dochodowości jest szeroko poruszany zarówno na łamach podręczników czy monografii akademickich, wydawnictw przeznaczonych dla praktyków rynkowych jak również w wielu publikacjach naukowych. Warto zauważyć, że liczba publikacji w języku polskim chociaż nie za wysoka systematycznie rośnie i~nie jest już tak niska jak jeszcze kilkanaście lat temu co zauważa \citet{swieton2002}.

W~niniejszej pracy szczególny nacisk położony jest na sprawdzenie w~warunkach polskich modelu zaprezentowanego przez \citet{dieboldli2005} będącego modyfikacją metody zastosowanej przez \citet{nelsonsiegelli1987} z~wykorzystaniem sposobu obliczeń przedstawionych w~\citet{dieboldiinni2005}. Niemniej jednak, w~dalszej części rozdziału przedstawiony zostanie zbiór pozostałych prac poruszających temat krzywej dochodowości. Z~uwagi na to, że tematyka niniejszej pracy odwołuje się do zagadnień dotyczących w dużej mierze rynku krajowego a~wyniki analiz często są charakterystyczne dla poszczególnych krajów, szczególny nacisk położony został na publikacje bądź to w~języku polskim bądź traktujące o polskich realiach.

W~jednej z~polskich publikacji \citet{marciniak2006} pisze o~sposobach wykorzystywanych w~Narodowym Banku Polskim służącym estymowaniu krzywych dochodowości. Przeprowadzane jest porównanie i~ocena pod kilkoma kryteriami modelu opartego na B-splajnie i~modelu pochodzącego ze \citet{svensson1994}. Zaprezentowane wyniki porównania dowodzą, że metoda B-splajnowa jest najbardziej niezawodna przy wykorzystaniu stabilizacji ze zmienną sankcją krzywizny (VRP - variable roughness penalty). W~dalszej części artykułu autor sugeruje zmianę sposobu implementacji powyższej metody w~celu uzyskania w~łatwiejszy sposób analitycznych rozwiązań problemu i~skrócenie czasu potrzebnego na obliczenia. Artykuł kończy się przedstawieniem uzyskanej krzywej dochodowości na danych polskich i~analizie jej dynamiki. Test wykorzystujący szeregi czasowe służy do oceny wpływu szoków na rynku papierów skarbowych.

W~polskich publikacjach i~materiałach akademickich autorzy bardzo często odwołują się do jednego z~pierwszych lokalnych obszernych opracowań dotyczących krzywej dochodowości autorstwa \citet{swieton2002}. Autor rozpoczyna swoją pracę od wprowadzenia teorii dotyczącej krzywej dochodowości jak i~oprocentowań obligacji i~innych papierów dłużnych. Następnie opisany jest początek i~rozwój rynku obligacji skarbowych w Polsce.



\chapter{Filtr Kalmana}
\label{sec:kalm}

Bieżący rozdział (czyli rozdział~\ref{sec:kalm}) zawiera odwołanie do samego siebie. Stosujmy odwołania automatyczne.

Dodatkowo w niniejszym rozdziale zamieszczono przykładową tabelę (uwaga --- \LaTeX\ umiejscawia ją dynamicznie).
\section{Podrozdział}
% warto w kodzie poniżej zwrócić uwagę na prawidłowy sposób formatowania liczb z przecinkiem jako znakiem dziesiętnym
% a dodatkowo na wyrównywanie liczb ułatwiające porównywanie


\begin{table}[h]
\centering
\caption{Przykładowa tabela.}
\label{tab:przyklad}
\footnotesize
\begin{tabular}{|l|r|r|}
\hline
Wariant & $N=5$ & $N=10$\\
\hline
A & $2$ & $3\phantom{{,}1}$ \\
B & $5$ & $3{,}1$ \\

\hline
\end{tabular} 
\end{table}

\clearpage

\chapter{Wyniki}
\label{sec:wyniki}

Wzory proste umieszczamy w tekście $a^2+b^2=c^2$, wzory bardziej skomplikowane --- poza nim:
\begin{equation}
\sum_{n=1}^{+\infty}\frac{1}{n^2}=\frac{\pi^2}{6}.
\end{equation}

\section{Pierwszy podrozdział}

\begin{figure}[h]
  \centering
  \begin{tikzpicture}[scale=4]
    \draw[->] (0,0) -- (1.4,0) node[anchor=west] {$n$};
    \draw[->] (0,0) -- (0,1.35) node[anchor=south] {Iloraz};
    \foreach \y in {0, 0.25,0.5,0.75,1, 1.25}
      \draw (0.03, \y) -- (-0.03, \y) node[anchor=east] {$\y$};
     \foreach \x in {0, 2, 4, 6, 8, 10, 12}
       \draw ({\x/10}, 0.03) -- ({\x/10}, -0.03) node[anchor=north] {$\x$};

       \draw (0.200000,1.164983) circle[radius=0.02];
       \draw (0.300000,1.176768) circle[radius=0.02];
       \draw (0.400000,0.877412) circle[radius=0.02];
       \draw (0.500000,0.456929) circle[radius=0.02];
       \draw (0.600000,0.351427) circle[radius=0.02];
       \draw (0.700000,0.294229) circle[radius=0.02];
       \draw (0.800000,0.246016) circle[radius=0.02];
       \draw (0.900000,0.224231) circle[radius=0.02];
       \draw (1.000000,0.168351) circle[radius=0.02];
       \draw (1.100000,0.191722) circle[radius=0.02];
       \draw (1.200000,0.173029) circle[radius=0.02];
       \draw (1.300000,0.176309) circle[radius=0.02];

       \fill (0.200000,1.225646) circle[radius=0.02];
       \fill (0.300000,0.986035) circle[radius=0.02];
       \fill (0.400000,0.945566) circle[radius=0.02];
       \fill (0.500000,0.688660) circle[radius=0.02];
       \fill (0.600000,0.556635) circle[radius=0.02];
       \fill (0.700000,0.483555) circle[radius=0.02];
       \fill (0.800000,0.443744) circle[radius=0.02];
       \fill (0.900000,0.413361) circle[radius=0.02];
       \fill (1.000000,0.344671) circle[radius=0.02];
       \fill (1.100000,0.374543) circle[radius=0.02];
       \fill (1.200000,0.358811) circle[radius=0.02];
       \fill (1.300000,0.362736) circle[radius=0.02];
  \end{tikzpicture}
  \caption{Oto przykładowy, skomplikowany rysunek. Często będzie łatwiej. \label{fig:algcomparison}}
\end{figure}

Każdy rysunek przywołujemy w tekście, także rysunek \ref{fig:algcomparison}. Można też ładować obrazy, jak pokazano na stronie tytułowej.

\section{Drugi}

\section{Trzeci}

\clearpage

\chapter{Podsumowanie}



\clearpage
\addcontentsline{toc}{chapter}{Bibliografia}
\begin{thebibliography}{99}
\setlength{\itemsep}{0pt}%
\bibitem[Choudhry~(2004)]{choudhry2004} Choudhry M.~(2004), Analysing \& Interpreting the Yield Curve, John Wiley and Sons (Asia) Pte Ltd
\bibitem[Rubaszek~(2012)]{rubaszek2012} Rubaszek M.~(2012), Modelowanie polskiej gospodarki z~pakietem R, Oficyna Wydawnicza SGH
\bibitem[Świętoń~(2002)]{swieton2002} Świętoń M. (2002), Terminowa struktura dochodowości skarbowych papierów wartościowych w~Polsce w~latach 1998-2001, \textit{Materiały i~Studia}, 150
\bibitem[Diebold i Li~(2005)]{dieboldli2005} Diebold F. X., Li C. (2005), Forecasting the term structure of government bond yields, \textit{Journal of Econometrics}, 130, s.~337-364
\bibitem[Nelson i Siegel~(1987)]{nelsonsiegelli1987} Nelson C. R., Siegel A. F. (1987), Parsimonious Modeling of Yield Curves, \textit{The Journal of Business}, 60, s.~473-489
\bibitem[Diebold i inni~(2005)]{dieboldiinni2005} Diebold F. X., Rudebusch G. D., Aruoba S. B. (2005), Forecasting the term structure of government bond yields, \textit{Journal of Econometrics}, 131, s.~309-338
\bibitem[Marciniak~(2006)]{marciniak2006} Marciniak M. (2006), Yield Curve Estimation at the National Bank of Poland. Spline Based Methods, Curve Smoothing and Market Dynamics, \textit{Bank i Kredyt}, 37, s.~52-74
\bibitem[Svensson~(1994)]{svensson1994} Svensson L. E. O. (2006), Estimating and Interpreting Forward Interest Rates: Sweden 1992-1994, \textit{National Bureau of Economic Research}, 4871


\end{thebibliography}

\clearpage
\addcontentsline{toc}{chapter}{Spis rysunków}
\listoffigures

\clearpage
\listoftables
\addcontentsline{toc}{chapter}{Spis tabel}

\appendix
\chapter*{Kody źródłowe}
\addcontentsline{toc}{chapter}{Kody źródłowe}

\section*{Analiza 1}
\begin{verbatim}
x <- 1:10
y <- x + rnorm(10)
summary(lm(y ~ x))
\end{verbatim}

\section*{Analiza 2}
\begin{verbatim}
x <- rnorm(10000)
plot(density(x))
\end{verbatim}

\clearpage

\chapter*{Streszczenie}
\addcontentsline{toc}{chapter}{Streszczenie}

Tutaj piszemy streszczenie. Między 200 a 350 słów. Nie jest omówieniem struktury pracy. Zawiera cel, metodę, dane, wyniki, wnioski. Nie ma odwołań źródłowych, list, wykresów, tabel. Ma być zrozumiałe dla osoby nieczytającej pracy.

\end{document}
